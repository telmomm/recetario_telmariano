\mysection{Contramuslos con miel}

\subsection*{INGREDIENTES}
\begin{itemize}
    \item 1 cebolla
    \item \nicefrac{1}{2} pimiento verde
    \item \nicefrac{1}{2} pimiento rojo
    \item 3 ajos
    \item 2 zanahorias
    \item 12 muslos de pollo
    \item 1 + \nicefrac{1}{2} vaso de vino blanco
    \item 10 contramuslos de pollo
    \item Miel
    \item Salsa de soja
    \item 1l de caldo de pollo
    \item 8 dientes de ajo
    \item Mantequilla
    \item Vinagre
\end{itemize}

\subsection*{PASOS}
\begin{enumerate}
    \item Dorar los muslos de pollo y retirar.
    \item Sofreír las verduras troceadas en la olla express.
    \item Añadir 1 + \nicefrac{1}{2} vaso de vino blanco, dejar reducir y añadir los muslos.
    \item Cerrar la olla y cocer durante 10 minutos. Separar los muslos de la olla, triturar las verduras y arreglar de sal.
    \item Dar un hervir a la salsa (3 minutos)
    \item Juntar con el pollo y hervir juntos otros 5 minutos.
    \item Sazonar los contramuslos de pollo con pimienta y sal.
    \item Sellar los contramuslos en la sartén con aceite y mantequilla.
    \item Dorar los ajos con el aceite de los contramuslos (no todo).
    \item Mezclar 330g de miel, con 65g de salsa de soja y 20g de vinagre (añadir sésamo a elección y arreglar de sal si la soja es muy dulce).
    \item Añadir el preparado anterior en una cazuela junto con los ajos y con \nicefrac{1}{2} l de caldo de pollo.
    \item Añadir el pollo a la cazuela y cocer a fuego medio durante 20 minutos (tapado). Ir regando los muslos con la salsa de vez en cuando.
    \item Destapar, arreglar de sal y dejar espesar a fuego medio durante 10 minutos (destapado).
    \item Si el pollo está hecho, retirar y espesar la salsa hasta conseguir la textura deseada (añadir 2-3 cucharadas de vinagre y 1 cucharada de maicena).
    \item Arreglar de sal y volver a juntar con el pollo.
\end{enumerate}