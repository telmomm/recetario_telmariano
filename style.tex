% Configuración de geometry para A5
\geometry{
  a5paper,
  left=20mm,
  right=20mm,
  top=20mm,
  bottom=20mm
}

% Configuración de fancyhdr
\pagestyle{fancy}
\fancyhf{}
\fancyhead[R]{\nouppercase{\rightmark}} % Nombre del capítulo en el encabezado derecho
\fancyfoot[C]{\thepage} % Número de página en el centro del pie de página
\renewcommand{\footrulewidth}{0.4pt} % Línea horizontal en el pie de página

% Redefinir \chaptermark para mostrar solo el nombre del capítulo
\renewcommand{\chaptermark}[1]{\markright{#1}}

% Redefinir el formato del capítulo para que no se muestre nada
\titleformat{\chapter}[display]
  {\normalfont\huge\bfseries}
  {}
  {0pt}
  {\vspace{-20em}} % Ajustar el espacio vertical para ocultar el título del capítulo
  
% Definir un nuevo comando para la portada del capítulo
\newcommand{\chaptercover}[3]{
    %\cleardoublepage % Asegurarse de que no haya páginas en blanco antes
    \phantomsection % Crear una sección fantasma para el hipervínculo en la TOC
    \addcontentsline{toc}{chapter}{#1} % Añadir el capítulo a la tabla de contenidos
    \chaptermark{#1} % Marcar el capítulo para los encabezados
    \thispagestyle{empty}
    \begin{center}
        \vspace*{5cm}
        {\Huge\bfseries #1\par}
        \vspace{1cm}
        {\Large #2\par}
        \vspace{1cm}
        {\large #3\par}
        \vfill
    \end{center}
    \clearpage
}

\newcommand{\mysection}[1]{
  \section*{#1}
  \phantomsection
  \addcontentsline{toc}{section}{#1}% Añadir la sección a la tabla de contenidos
}


